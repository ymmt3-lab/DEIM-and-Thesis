\documentclass[a4paper,12pt,oneside,openany,autodetect-engine,dvipdfmx,platex]{jsreport}
\usepackage{latexsym}
\usepackage{booktabs}
\usepackage{color}
\usepackage{colortbl}
\usepackage{graphicx}
\usepackage{amsmath}
\usepackage{bm}
\usepackage{multirow}
\usepackage{hyperref}
\usepackage{pxjahyper}

\newcommand{\ChangeHeaderRule}{%
     \let\subsubsection\subsection
     \let\subsection\section
     \let\section\chapter
}

%%%%%%%%%%%%%%%%%%%%%%%%%%%%
% 表紙
%%%%%%%%%%%%%%%%%%%%%%%%%%%%
\begin{document}

\begin{titlepage}
\begin{flushright}
%{\large
指導教員(主査):山本祐輔 講師\\
副査:XXX 教授
%}
\end{flushright}
\begin{center}
\vspace*{80pt}
{\large 2018年度 静岡大学情報学部 卒業論文}\\

\vspace*{40pt}
{\huge XXXに関する研究}\\
\vspace{10pt}
{\Large --- サブタイトル ---}\\
\vfill
{静岡大学 情報学部 ISプログラム 所属}\\
{学籍番号 XXXXX}\\
\vspace{20pt}
{静岡 花子}\\
\vspace{20pt}
{\today}\\
\end{center}
\vfill
\end{titlepage}


%%%%%%%%%%%%%%%%%%%%%%%%%%%%
% アブストラクト(300字程度)
%%%%%%%%%%%%%%%%%%%%%%%%%%%%
\begin{abstract}
これはアブストラクトです.300字程度でまとめてください.

\end{abstract}

%%%%%%%%%%%%%%%%%%%%%%%%%%%%
% 目次
%%%%%%%%%%%%%%%%%%%%%%%%%%%%

\tableofcontents

\newpage
\listoffigures
\listoftables


% 必ず奇数ページからはじまるように改ページする
%\cleardoublepage






%%%%%%%%%%%%%%%%%%%%%%%%%%%%
% 本文
%%%%%%%%%%%%%%%%%%%%%%%%%%%%
\ChangeHeaderRule

% 0章:LateXの使い方
\section{LaTeXの使い方}
本章ではLaTeXの使い方をちょっとだけ解説します。
LaTeXは内容とスタイル(見た目)を切り分けて文書を編集することができるソフトウェアです。
コマンドを駆使して,美しい文書を作成することができます。

\subsection{LaTeX環境}
環境構築が嫌いな人は\href{https://ja.overleaf.com}{OverLeaf}を使いましょう。
OverLeafはオンライン上でLaTeXを執筆できる環境です。
自分のPC/Macの環境を汚さない,環境構築に苦労しないというメリットがあります。
一方,インターネットに接続していないと執筆作業ができないというデメリットがあります。

環境構築に抵抗がない人は自分のPC/Mac上にLaTeX環境を構築しましょう。
世の中には様々なLaTeX環境があります。
最も有名なのはTeXLiveです。
こだわりがなければ\href{https://texwiki.texjp.org/?TeX%20Live%2FWindows}{TeXLiveのウェブサイト}からソフトウェアをダウンロードしインストールしましょう。
インストールが完了したらTeXWorksというアプリケーションを起動してください。
このアプリケーションを使うことで,LaTeXで文書を作成することができます。

\subsection{見出し}
文章を構造化するには,内容を章別,項別に整理することが重要です。
例えばこの文書では「第1章 LaTeXの使い方」が章に対応し,「1.1 LaTeX環境」が節に対応します。
LaTeXでは\texttt{section}コマンドを用いることで章見出しを,\texttt{subsection}コマンドを用いることで節見出しを作成することができます。
実際の使い方については,\texttt{contents/text}ディレクトリにある\href{https://github.com/ymmt3-lab/DEIM-and-Thesis/blob/master/contents/text/latex.tex}{\texttt{latex.tex}ファイル}の中身を覗いてみてください。
本章に対応するLaTeXソースが確認できます。


\subsection{文字装飾}

\subsection{箇条書き}

\subsection{図}

\subsection{表}

\subsection{引用}


% 1章:はじめに
\section{はじめに}
これは「はじめに」です.


% 2章:関連研究
\section{関連研究}
本章では,関連研究について記す.


% 3章:提案事項
\section{提案内容(このセクション名は内容に応じて変更)}
本章では,●●を行うためのxxxの方法について述べる.


% 4章:評価実験
\section{評価実験}\label{sec:experiment}
本章では,提案手法に関する評価実験について記す.


% 5章:結果
\section{結果}\label{sec:results}
本章では,\ref{sec:experiment}で述べた実験の結果について記す.


% 6章:考察
\section{考察}
本章では,\ref{sec:results}で記した結果について考察を行う.


% 7章:おわりに
\section{おわりに}
本稿では,●●を行うためのxxxの方法についての提案を行った.




%\vspace{30mm} <- 文献が本文と近すぎるときは適宜利用してください.
\vspace{2em}

\begin{thebibliography}{99}
\bibitem{Codd1970}
  E. F. Codd,
  ``A Relational Model of Data for Large Shared Data Banks,''
  Communications of the {ACM} (CACM), Vol. 13, No. 6, pp. 377--387, 1970.
\end{thebibliography}


\section*{謝辞}

本研究の遂行ならびに論文の作成にあたり,ご指導を賜りましたXX大学XX先生に謹んで深謝の意を表します.

本論文をまとめるにあたり,副査として有益な御助言と御教示を賜りましたXX先生に心より感謝の意を表します.

本研究の遂行ならびに論文の作成にあたり御協力いただいた,XX大学XX研究室の皆様に感謝致します.特に,XXしてくれたXX君に心より感謝致します.

最後に,これまで暖かく見守ってくれた両親に感謝します.


\begin{flushright}
20XX年3月 静岡 花子
\end{flushright}

\end{document}
