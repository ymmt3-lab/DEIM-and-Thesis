\section{LaTeXの使い方}
本章ではLaTeXの使い方をちょっとだけ解説します。
LaTeXは内容とスタイル(見た目)を切り分けて文書を編集することができるソフトウェアです。
コマンドを駆使して,美しい文書を作成することができます。

\subsection{LaTeX環境}
環境構築が嫌いな人は\href{https://ja.overleaf.com}{OverLeaf}を使いましょう。
OverLeafはオンライン上でLaTeXを執筆できる環境です。
自分のPC/Macの環境を汚さない,環境構築に苦労しないというメリットがあります。
一方,インターネットに接続していないと執筆作業ができないというデメリットがあります。

環境構築に抵抗がない人は自分のPC/Mac上にLaTeX環境を構築しましょう。
世の中には様々なLaTeX環境があります。
最も有名なのはTeXLiveです。
こだわりがなければ\href{https://texwiki.texjp.org/?TeX%20Live%2FWindows}{TeXLiveのウェブサイト}からソフトウェアをダウンロードしインストールしましょう。
インストールが完了したらTeXWorksというアプリケーションを起動してください。
このアプリケーションを使うことで,LaTeXで文書を作成することができます。

\subsection{見出し}
文章を構造化するには,内容を章別,項別に整理することが重要です。
例えばこの文書では「第1章 LaTeXの使い方」が章に対応し,「1.1 LaTeX環境」が節に対応します。
LaTeXでは\texttt{section}コマンドを用いることで章見出しを,\texttt{subsection}コマンドを用いることで節見出しを作成することができます。
実際の使い方については,\texttt{contents/text}ディレクトリにある\href{https://github.com/ymmt3-lab/DEIM-and-Thesis/blob/master/contents/text/latex.tex}{\texttt{latex.tex}ファイル}の中身を覗いてみてください。
本章に対応するLaTeXソースが確認できます。


\subsection{文字装飾}

\subsection{箇条書き}

\subsection{図}

\subsection{表}

\subsection{引用}
